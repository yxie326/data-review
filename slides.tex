\documentclass{beamer}

\mode<presentation>
{
    % Include the theme
    \usetheme{gatech}
    \setbeamercovered{transparent = 28}
}

\usepackage{braket}

\title{Data Review}

\author{Yi Xie}
\GTtoc{Table of contents}

\begin{document}

\GTtitle

\section{Overview}

    \begin{frame}{Intermolecular Energies}
        \begin{itemize}
            \item Supermolecular approach 
                $$E_{int} = E_{AB} - E_A - E_B$$
                \begin{itemize}
                    \item Straightforward, but cannot separate different types of interactions
                    \item Can adopt to different electronic structure methods
                    \item DFT-D3 with proper functional can be both cheap and accurate
                \end{itemize}
            \item Symmetry-Adapted Perturbation Theory
            \begin{itemize}
                \item Can give details about different types of interactions; important in understanding their nature 
                \item Not as cheap as DFT-D3
                \item SAPT0 is somewhat cheap, but does not include intramonomer correlation
            \end{itemize}
        \end{itemize}
    \end{frame}

    \begin{frame}{SAPT(DFT)}
        \begin{itemize}
            \item Attempt to inlude intramonomer correlation in a cheap way
            \item Replaces HF orbitals with KS orbitals
            \item Needs to consider orbital response for dispersion terms
            \item Exchange-dispersion term needs to be estimated from scaling 
            \item Investigate the accuracy and efficiency of SAPT(DFT)
        \end{itemize}
    \end{frame}

    \begin{frame}{Three-Body Interaction}
        \begin{itemize}
            \item Crucial in computing lattice energies
            \item DFT-D3 does not perform well for three-body interaction
            \item MP2.5 scales as $O(N^6)$, MP2 $O(N^5)$ but lacks three-body dispersion
            \item Three-body dispersion can be implemented with SAPT(DFT) in $O(N^5)$
        \end{itemize}   
    \end{frame}

\section{SAPT(DFT) Implementation}
    
    \subsection{Theory}
             
        \begin{frame}{Idea of SAPT(DFT)}
            \begin{itemize}
                \item SAPT energy in orders of interaction and fluctuation potentials; $n$ denotes order in $V$ and $k,l$ for $W_A, W_B$
                $$H = F_A + F_B + V + W_A + W_B$$
                $$E_{int} = \sum_{n=1}^\infty \sum_{k=0}^\infty \sum_{l=0}^\infty \left(E_{pol}^{(nkl)} + E_{exch}^{(nkl)}\right)$$
                \item SAPT0: $n = 2, k = l = 0$, no intramonomer correlation, $O(N^5)$ cost
                \item Many-body SAPT: $k,l \geq 2$, $O(N^7)$ or higher cost
                \item SAPT(DFT): Use Kohn-Sham operator $K_{A,B}$ instead of Fock operator $F_{A,B}$, $O(N^5)$ cost
                \item Primitive SAPT(DFT) works well on 1st-order terms but not 2nd-order terms (especially dispersion terms), needs orbital response for them
            \end{itemize}
        \end{frame}

        \begin{frame}{Dispersion Term}
            \begin{eqnarray}
                E_{disp}^{(2)} &= -\sum_{m \neq 0, n \neq 0}\,\frac{\left| \braket{\Psi^A_0 \Psi^B_0|V_{AB}|\Psi^A_m \Psi^B_n}\right|^2}{E^A_m - E^A_0 + E^B_n - E^B_0} \\
                &= -4\sum_{ia \in A,jb \in B} \frac{\left|\left(i^Aa^A|j^Bb^B\right)\right|^2}{\epsilon_a^A-\epsilon_i^A+\epsilon_b^B-\epsilon_j^B}
            \end{eqnarray}
            \begin{itemize}
                \item ALDA kernel good for pure GGA functional but not for hybrid functional 
                \item Exact exchange in $v_{xc} \rightarrow$ increased $\epsilon_{ij}^{ab} \rightarrow$ decreased $E_{disp}^{(2)}$ 
                \item Either hybrid ALDA kernel or localized HF (LHF) exchange to compensate
                $$f_{xc} = \alpha f_{xc}^{HF} + (1-\alpha) f_{xc}^{ALDA}$$
            \end{itemize}            
        \end{frame}


        \begin{frame}{Test 5}

            \begin{tabular}{|c|c|c|}
            \hline
            \textbf{Test0} & \textbf{Test1} & \textbf{Test2} \\
            \hline
            Test1 & Test2 & Test3 \LaTeX  \\
            \hline
            Test4 & Test5& Test6\\
            \hline
            \end{tabular}
            
        \end{frame}

        \begin{frame}{Test 6}

            \begin{exampleblock}{Example block}
                \smallskip 

                \begin{tabular}{|c|c|c|}
                \hline
                \textbf{Test0} & \textbf{Test1} & \textbf{Test2} \\
                \hline
                Test1 & Test2 & Test3 \LaTeX  \\
                \hline
                Test4 & Test5& Test6\\
                \hline
                \end{tabular}
            \end{exampleblock}

            \begin{block}{Block}
                \smallskip 

                \begin{tabular}{|c|c|c|}
                \hline
                \textbf{Test0} & \textbf{Test1} & \textbf{Test2} \\
                \hline
                Test1 & Test2 & Test3 \LaTeX  \\
                \hline
                Test4 & Test5& Test6\\
                \hline
                \end{tabular}
            \end{block}
        \end{frame}

	 \begin{frame}{Test 7}
            
         	\begin{block}{Block A}
            
                \begin{description}[longest label]
                    \item[short] Short stuff
                    \item[long] Longer stuff
                    \item[longest label] Longest stuff (insert cat)
                \end{description}
            
                \begin{itemize}
                    \item item1
                    \item item2
                    \item item3
                \end{itemize}  
            \end{block}

        \end{frame}

        \begin{frame}[t]{Frame with Columns}
            \begin{columns}[t]
                \column{0.4\textwidth}

                \begin{block}{Block 1}
                    Text here
                \end{block}

                \column{0.4\textwidth}

                \begin{block}{Block 2}
                    More text here
                \end{block}
            \end{columns}
        \end{frame}

        \begin{frame}[t]{Frame without Columns}
            \leavevmode
            \begin{block}{Block}
                Even more text here
            \end{block}
        \end{frame}
\end{document}
