\documentclass{beamer}

\mode<presentation>
{
    % Include the theme
    \usetheme{gatech}
    \setbeamercovered{transparent = 28}
}

\usepackage{braket}
\usepackage[labelformat=empty]{caption}
\usepackage{siunitx}

\newcommand\blfootnote[1]{%
  \begingroup
  \renewcommand\thefootnote{}\footnote{#1}%
  \addtocounter{footnote}{-1}%
  \endgroup
}

\title{Implementation and Application of Density Functional Theory based Symmetry-Adapted \linebreak
Perturbation Theory for Dimers, Trimers and Molecular Crystals}

\author{Yi Xie}
\date{July 25, 2022}

\begin{document}

\GTtitle

\section{SAPT(DFT): Background}

    \begin{frame}{Noncovalent Interaction}
        \begin{itemize}
            \item Phase transition, stability of crystal structure
            \item Drug binding, DNA/RNA/protein structure
            \item Many-body Expansion for the energy of complex system:
                $$E = \sum_A E_A + \sum_{AB} E^{\rm{int,2}}_{AB} + \sum_{ABC} E^{\rm{int,3}}_{ABC} +  \cdots$$
        \end{itemize} 
    \end{frame}

    \begin{frame}{Intermolecular Energies}
        \begin{itemize}
            \item Supermolecular approach 
                $$E^{\rm{int,2}}_{AB} = E_{AB} - E_A - E_B$$
                $$E^{\rm{int,3}}_{ABC} = E_{ABC} - E_{AB} - E_{AC} - E_{BC} + E_A + E_B + E_C$$
            \item Symmetry-Adapted Perturbation Theory (SAPT)
                $$E^{\rm{int,2}} = E_{\rm{elst}}^{(1)} + E_{\rm{exch}}^{(1)} + E_{\rm{ind}}^{(2)} + E_{\rm{exch-ind}}^{(2)} + E_{\rm{disp}}^{(2)} + E_{\rm{exch-disp}}^{(2)}$$
        \end{itemize}
    \end{frame}

    \begin{frame}{SAPT(DFT)}
        \begin{itemize}
            \item Hamiltonian partitioning:
                $$H = F_A + F_B + V_{AB} + W_A + W_B$$ 
                $$H = K_A + K_B + V_{AB}$$ 
            \item ``Uncoupled'' sum-over-states approximation of $E_{\rm{disp}}^{(2)}$ and in terms of frequency-dependent density susceptibility (FDDS):
                \begin{eqnarray}
                    \nonumber
                    E_{\rm{disp,u}}^{(2)} &=& -4\sum_{ar \in A,bs \in B} \frac{\left|\left(ar|bs\right)\right|^2}{\epsilon_{ab}^{rs}}\\ \nonumber
                    &=& -\frac{1}{2\pi}\int_0^\infty d\omega \int d\mathbf{r}_A d\mathbf{r}'_A d\mathbf{r}_B d\mathbf{r}'_B \\ \nonumber
                    & & \frac{1}{|\mathbf{r}_A-\mathbf{r}_B|} \frac{1}{|\mathbf{r}'_A-\mathbf{r}'_B|} \chi_0^A\left(\mathbf{r}_A,\mathbf{r}'_A|i\omega\right)\chi_0^B\left(\mathbf{r}_B,\mathbf{r}'_B|i\omega\right)
                \end{eqnarray}
        \end{itemize}
    \end{frame}

    \begin{frame}{Uncoupled $E_{\rm{disp}}^{(2)}$}
        \begin{figure}
            \centering
            \includegraphics[width=0.5\textwidth]{uncoupled.png}
            \caption{Mean ($\delta$) and mean absolute ($\epsilon$) percentage deviations of uncoupled (u) and coupled (c) $E_{\rm{disp}}^{(2)}$ from SAPT2+ results.\blfootnote{A. He{\ss}elmann and G. Jansen, Chem. Phys. Lett. \textbf{367}, 778 (2003).}}
        \end{figure}
    \end{frame}

    \begin{frame}{Coupled $E_{\rm{disp}}^{(2)}$}
        \begin{itemize}
            \item Replacing uncoupled FDDS with coupled FDDS, solved from the coupled Kohn--Sham (CKS) TDDFT equation:\blfootnote{M. Pito{\v{n}}{\'{a}}k and A. Hesselmann, J. Chem. Theory Comput. \textbf{6}, 168 (2010).}
            $$\boldsymbol{\chi} = \boldsymbol{\chi}_0 + \boldsymbol{\chi}_0 \mathbf{S}^{-1} \mathbf{W} \left( \mathbf{S} - \boldsymbol{\chi}_0 \mathbf{S}^{-1} \mathbf{W} \right)^{-1} \boldsymbol{\chi}_0$$
            \item Exchange-correlation kernel term in $\mathbf{W}$ approximated by adiabatic local-density approximation (ALDA) kernel:
            \begin{eqnarray}
                \nonumber
                W_{PQ} &=& \left( P | r_{12}^{-1} | Q \right) + \left( P | f_{\rm{xc}} | Q \right) \\ \nonumber
                & \approx & \left( P | r_{12}^{-1} | Q \right) + \left( P | f_{\rm{xc}}^{\rm{ALDA}} | Q \right)
            \end{eqnarray}
        \end{itemize}
    \end{frame}

    \begin{frame}{Hybrid Functional}
        \begin{itemize} 
            \item Local Hartree--Fock (LHF) approach  
            \begin{itemize}
                \item Computing LHF potential in each KS SCF iteration
                \item $O(N^4)$ with very large constant factor
                \item Different set of KS orbitals with smaller occupied--virtual gap
            \end{itemize}
            \item Hybrid ALDA kernel
            \begin{itemize}
                \item Mixing CHF and CKS equations to solve for FDDS
                \item CKS involves integral of form $(ar|a'r')$, $O(N^4)$ with density fitting
                \item CHF involves $(aa'|rr')$ and $(ar'|a'r)$, $O(N^5)$
            \end{itemize}
        \end{itemize} 
    \end{frame}
        
    \begin{frame}{Coupled $E_{\rm{exch-disp}}^{(2)}$}
        \begin{itemize}
            \item Scaling from scaling uncoupled $E_{\rm{exch-disp}}^{(2)}$
            $$\tilde{E}^{(2)}_{exch-disp,r} = E^{(2)}_{exch-disp,u} \cdot \frac{E^{(2)}_{disp,r}}{E^{(2)}_{disp,u}}$$
            \item Fixed scaling factor from fitting S22$\times$5\blfootnote{A. He{\ss}elmann and T. Korona, J. Chem. Phys. \textbf{141}, 094107 (2014).}
            $$\tilde{E}^{(2)}_{exch-disp,r} = \alpha \cdot E^{(2)}_{exch-disp,u} (\alpha = 0.686)$$
            \item Value of $\alpha$ above fitted from $E_{exch-disp,u}^{(2)}$ \textit{with LHF orbitals}
        \end{itemize}
    \end{frame}

    \begin{frame}{S22 Dimer Set}
        \begin{figure}
            \centering
            \includegraphics[width=0.8\textwidth]{S22.png}
        \end{figure}
    \end{frame}

\section{SAPT(DFT): Implementation}

    \begin{frame}{Coupled FDDS with Hybrid Kernel}
        \begin{itemize}
            \item Recall coupled FDDS for pure ALDA kernel:
            $$\boldsymbol{\chi} = \boldsymbol{\chi}_0 + \boldsymbol{\chi}_0 \mathbf{S}^{-1} \mathbf{W} \left( \mathbf{S} - \boldsymbol{\chi}_0 \mathbf{S}^{-1} \mathbf{W} \right)^{-1} \boldsymbol{\chi}_0$$
            \item Coupled FDDS for hybird ALDA kernel, with $(aa'|rr')$ and $(ar'|a'r)$ contributions in $\boldsymbol{\chi}_0'$ and $\mathbf{K}'$:
            $$\boldsymbol{\chi} = \boldsymbol{\chi}_0' + \left( \boldsymbol{\chi}_0' \mathbf{S}^{-1} \mathbf{W} + \mathbf{K}' \right) \left[ \mathbf{S} - \left( \boldsymbol{\chi}_0' \mathbf{S}^{-1} \mathbf{W} + \mathbf{K}' \right) \right]^{-1} \boldsymbol{\chi}_0'$$
            \item Dispersion energy from integration over $\omega$:
            $$E_{\rm{disp,r}}^{(2)} =  -\frac{1}{2\pi}\int_0^\infty d\omega \mathrm{Tr}\left(\mathbf{S}^{-1}\boldsymbol{\chi}^A\mathbf{S}^{-1}\boldsymbol{\chi}\right)$$
        \end{itemize}
    \end{frame}
        
    \begin{frame}{Refitting $E_{\rm{exch-disp,r}}^{(2)}$ for non-LHF Orbitals}
        \begin{itemize}
            \item LHF vs. non-LHF orbitals: Only affects uncoupled second-order terms like $E_{\rm{disp,u}}^{(2)}$ and $E_{\rm{exch-disp,u}}^{(2)}$
            \item Similar $E_{\rm{disp,r}}^{(2)}$ for LHF + pure ALDA vs. non-LHF + hybrid ALDA, expect the same for $E_{\rm{exch-disp,r}}^{(2)}$
            \item Can fit non-LHF $E_{\rm{disp,u}}^{(2)}$ to LHF + pure ALDA $E_{\rm{disp,r}}^{(2)}$:
            \begin{eqnarray}
                \nonumber
                E_{\rm{disp,r}}^{(2)}(hybrid) & \approx & E_{\rm{disp,r}}^{(2)}(LHF) \\ \nonumber
                & \approx & \alpha \cdot E^{(2)}_{exch-disp,u}(non-LHF)
            \end{eqnarray}
            \item Fit for $\alpha$ using S22$\times$5 and test with S66$\times$8
        \end{itemize}
    \end{frame}

    \begin{frame}{SAPT Terms: LHF vs non-LHF Orbitals}
        \begin{figure}
        \centering
        \includegraphics[width=0.8\textwidth]{S66-terms.png}
        \caption{Hybrid vs. LHF values in kcal/mol for each term for S66 data set: (a) $E_{\rm{elst}}^{(1)}$, (b) $E_{\rm{exch}}^{(1)}$, (c) $E_{\rm{ind}}^{(2)}$, (d) $E_{exch-ind}^{(2)}$, (e) $E_{disp,r}^{(2)}$, (f) $E_{disp,u}^{(2)}$, (g) $E_{exch-disp,u}^{(2)}$} 
        \end{figure}        
    \end{frame}

    \begin{frame}{Fitting: S22$\times$5}
        \begin{figure}
            \centering
            \includegraphics[width=0.55\textwidth]{S22x5-exchdisp.png}
        \end{figure}   
    \end{frame}

    \begin{frame}{Testing: S66$\times$8}
        \begin{figure}
            \centering
            \includegraphics[width=0.55\textwidth]{S66x8-exchdisp.png}
        \end{figure}   
    \end{frame}

\section{SAPT(DFT): Results}

    \begin{frame}{Computational Approach}
        \begin{itemize}
            \item Benchmarking with S66 dimer set
            \item Using SAPT2+3(CCD)$\delta$MP2/aug-cc-pVTZ as reference
            \item Compare electrostatics, exchange, induction, dispersion terms and total IE
            \item Using aug-cc-pVDZ for SAPT0 and SAPT2+, aug-cc-pVTZ for other methods 
        \end{itemize}
    \end{frame}

    \begin{frame}{S66 Results}
        \begin{figure}
            \centering
            \includegraphics[width=0.8\textwidth]{S66_1.png}
        \end{figure}
    \end{frame}

    \begin{frame}{S66 Results (cont.)}
        \begin{figure}
            \centering
            \includegraphics[width=0.8\textwidth]{S66_2.png}
        \end{figure}
    \end{frame}

    \begin{frame}{Timing Systems}
        \begin{figure}
            \centering
            \includegraphics[width=0.8\textwidth]{timingsystems.png}
            \caption{Dimer systems for timing: (a) Watson-Crick adenine-thymine complex, (b) RDX dimer, (c) C$_{\textrm{60}}$--buckycather complex.}
        \end{figure}
    \end{frame}

    \begin{frame}{Watson-Crick Adenine-Thymine}
        \begin{figure}
            \centering
            \includegraphics[width=0.8\textwidth]{wcat.png}
        \end{figure}
    \end{frame}

    \begin{frame}{RDX Dimer}
        \begin{figure}
            \centering
            \includegraphics[width=0.8\textwidth]{RDX.png}
        \end{figure}
    \end{frame}

    \begin{frame}{C$_{\textrm{60}}$--Buckycatcher Complex}
        \begin{figure}
            \centering
            \includegraphics[width=0.6\textwidth]{C60_buckycatcher.png}
        \end{figure}
    \end{frame}

\section{SAPT(DFT): Conclusions}

    \begin{frame}{Conclusions}
       \begin{itemize}
            \item Implemented SAPT(DFT) using hybrid xc kernel, applicable to up to 3000 basis functions
            \item Scaling factor of $E_{\rm{exch-disp}}^{(2)}$ determined to be 0.770
            \item Accuracy of SAPT(DFT) comparable to SAPT2+ which scales as $O(N^7)$
            \item Iterative $O(N^4)$ terms not negligible; even dominating for smaller systems
       \end{itemize}
    \end{frame}

\section{Three-Body Dispersion: Background}

    \begin{frame}{Three-Body Interaction Energies}
        \begin{itemize}
            \item Cheapest conventional methods to model three-body interaction energies scale as $O(N^6)$ [MP2.5, SCS(MI)-CCSD]
            \item MP2 lacks three-body dispersion interactions
            \item Combine supermolecular MP2 with three-body dispersion correction term 
            \item Reduces computational cost for three-body interaction energies to the level of $O(N^5)$
        \end{itemize}
    \end{frame}

    \begin{frame}{CKS FDDS Dispersion}
        \begin{itemize}
            \item Two-body FDDS dispersion term in SAPT(DFT):
            $$E_{\rm{disp,r}}^{(2)} =  -\frac{1}{2\pi}\int_0^\infty d\omega \mathrm{Tr}\left(\mathbf{S}^{-1}\boldsymbol{\chi}^A\mathbf{S}^{-1}\boldsymbol{\chi}\right)$$
            \item Generalize to three-body case:
            $$E_{\rm{disp,r}}^{(3)} = \int_0^\infty d\omega\,\mathrm{Tr}\left(\mathbf{S}^{-1}\boldsymbol{\chi}^A\mathbf{S}^{-1}\boldsymbol{\chi}^B\mathbf{S}^{-1}\boldsymbol{\chi}^C\right)$$
        \end{itemize}
    \end{frame}

    \begin{frame}{Axilrod--Teller--Muto Dispersion}
        \begin{itemize}
            \item Dispersion energy of atom triplet:
            $$E_{\rm{ATM}}^{abc} = C_9^{abc} \frac{1 + 3{\rm{cos}} \theta_a {\rm{cos}} \theta_b {\rm{cos}} \theta_c}{(R_{ab}R_{bc}R_{ca})^3}$$
            $$C_9^{abc} \approx \sqrt{C_6^{ab} C_6^{bc} C_6^{ca}}$$
            \item Sum over atom triplets:
            $$E_{\rm{ATM}}^{ABC} = \sum_{a \in A} \sum_{b \in B}, \sum_{c \in C} f^{abc}_9 E_{\rm{ATM}}^{abc}$$
        \end{itemize}
    \end{frame}

    \begin{frame}{Empirical Damping}
        \begin{itemize}
            \item Tang--Toennies damping function:
            $$f^{abc}_9 = f^{ab}_6(R_{ab}, \beta) f^{ac}_6(R_{ab}, \beta) f^{ca}_6(R_{ab}, \beta)$$
            $$f_6(R, \beta) = 1 - \sum_{k=0}^{6} \left( \frac{\left(\beta R\right)^k}{k!} \right) e^{-\beta R}$$
            \item Chai--Head-Gordon damping function:
            $$f^{abc}_9 = \frac{1}{1 + 6 \left(\bar{R}_{abc}\right)^{-16}}$$
            $$\bar{R}_{abc} = \left(R_{ab} R_{bc} R_{ca} / R^{ab}_{0,BJ} R^{ac}_{0,BJ} R^{ca}_{0,BJ} \right)^{1/3}$$
        \end{itemize}
    \end{frame}


\section{Three-Body Dispersion: Results}

    \begin{frame}{Geometric Parameters}
        \begin{itemize}
            \item Define geometrical parameters based on intermolecular distances
            \item $R_{\rm{min}}$: Smallest of three closest-contact distances
            \item $R_{\rm{max}}$: Largest of three closest-contact distances
            \item Analyze growth of three-body contribution to crystal lattice energy with respect to $R_{\rm{min}}$ and $R_{\rm{max}}$
        \end{itemize}
    \end{frame}

    \begin{frame}{Geometric Parameters}
        \begin{figure}
            \centering
            \includegraphics[width=0.55\textwidth]{geom_param.png}
        \end{figure}
    \end{frame}

    \begin{frame}{Computational Methods}
        \begin{itemize}
            \item Crystalline benzene, carbon dioxide and triazine
            \item Using CCSD(T) with focal point approach as reference
            $$E = E^{\rm{HF}}(\rm{aQZ}) + \Delta E^{\rm{MP2}}(\rm{aTZ/aQZ}) + \Delta E^{\rm{CCSD(T)}}(\rm{aDZ}) $$
            \item Compare three-body contribution to crystal lattice energies with various methods: MP2, MP2+ATM(undamped/TT/CHG), MP2+CKS, MP2.5, SCS(MI)-CCSD, CCSD(T)
        \end{itemize}
    \end{frame}

    \begin{frame}{$R_{\rm{min}}$ Dependence: Benzene}
        \begin{figure}
            \centering
            \includegraphics[width=0.55\textwidth]{benzene_Rmin.png}
        \end{figure}
    \end{frame}

    \begin{frame}{$R_{\rm{min}}$ Dependence: Carbon Dioxide}
        \begin{figure}
            \centering
            \includegraphics[width=0.55\textwidth]{carbon_dioxide_Rmin.png}
        \end{figure}
    \end{frame}

    \begin{frame}{$R_{\rm{min}}$ Dependence: Triazine}
        \begin{figure}
            \centering
            \includegraphics[width=0.55\textwidth]{triazine_Rmin.png}
        \end{figure}
    \end{frame}

    \begin{frame}{$R_{\rm{max}}$ Dependence: Benzene}
        \begin{figure}
            \centering
            \includegraphics[width=0.55\textwidth]{benzene_Rmax.png}
        \end{figure}
    \end{frame}

    \begin{frame}{$R_{\rm{max}}$ Dependence: Carbon Dioxide}
        \begin{figure}
            \centering
            \includegraphics[width=0.55\textwidth]{carbon_dioxide_Rmax.png}
        \end{figure}
    \end{frame}

    \begin{frame}{$R_{\rm{max}}$ Dependence: Triazine}
        \begin{figure}
            \centering
            \includegraphics[width=0.55\textwidth]{triazine_Rmax.png}
        \end{figure}
    \end{frame}
    
\section{Three-Body Dispersion: Conclusions}

    \begin{frame}{Conclusions}
       \begin{itemize}
            \item Three-body crystal lattice energies converge quickly as $R_{\rm{min}}$ increases, less smooth for $R_{\rm{max}}$ but can be cutoff at 10~\AA
            \item Three-body dispersion contributions crucial to lattice energies for crystals studied
            \item CKS FDDS dispersion overestimates three-body dispersion, likely missing exchange-dispersion terms
            \item Performance of Tang-Toennies damped ATM comparable to MP2.5
            \item Can use CCSD(T) for trimers with small $R_{\rm{min}}$ and MP2+ATM(TT) for larger ones
       \end{itemize}
    \end{frame}

\section{Acknowledgment}

    \begin{frame}{Acknowledgment}
        \begin{figure}
            \centering
            \includegraphics[width=0.7\textwidth]{group_21Jul.jpg}
        \end{figure}
    \end{frame}
    
        %\begin{frame}{GRAC}
        %    \begin{itemize}
        %        \item (Hybrid-)GGA functionals does not have correct long-range behavior $v_{xc}(r) \rightarrow -1/r + (I_p + \epsilon_{\text{HOMO}})$
        %        \item Underestimates o-v gap as a consequence
        %        \item Functionals like LB94 have correct asymptotic behavior, but poor in bulk region
        %        \item Using gradient-regulated asymptotic correction (GRAC) scheme to connect PBE0 and LB94 with the switching function $f[g(\boldsymbol{r})]$:
        %        $$v_{xc}^{\text{GRAC}} = \left\{1-f[g(\boldsymbol{r})]\right\}v_{xc}^{\text{PBE0}} + f[g(\boldsymbol{r})] v_{xc}^{\text{LB94}}$$
        %        $$f[g(\boldsymbol{r})] = \left( 1+e^{-\alpha [g(\boldsymbol{r}) - \beta]} \right)^{-1}$$
        %        $$g(\boldsymbol{r}) = \frac{\left| \nabla \rho (\boldsymbol{r}) \right|}{\rho^{4/3}(\boldsymbol{r})}$$
        %    \end{itemize}
        %\end{frame}        
        %
        %\begin{frame}{GRAC \& Long-Range Behavior}
        %    \begin{figure}
        %    \centering
        %    \includegraphics[width=0.65\textwidth]{GRAC.png}
        %    \caption{Radial densities $r^2\rho(r)$ of Ne atom (right) and errors compared to CCSD(T) density (left) for various xc potentials.\footnote{G. Jansen, WIREs Comput. Mol. Sci. \textbf{4}, 127 (2014).}} 
        %    \end{figure}
        %\end{frame}


\end{document}
