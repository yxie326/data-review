\documentclass{beamer}

\mode<presentation>
{
    % Include the theme
    \usetheme{gatech}
    \setbeamercovered{transparent = 28}
}

\title{Data Review}

\author{Yi Xie}
\GTtoc{Table of contents}

\begin{document}

\GTtitle

\section{Introduction}

    \subsection{Overview}

		\begin{frame}{Theoretical Methods for Intermolecular Interaction}
		  \begin{itemize}
		      \item Supermolecular approach 
		        $$E_{int} = E_{AB} - E_A - E_B$$
		        \begin{itemize}
		            \item Straightforward, but cannot separate different types of interactions
		            \item Can adopt to different electronic structure methods
					\item DFT-D3 with proper functional can be both cheap and accurate
		        \end{itemize}
		      \item Symmetry-Adapted Perturbation Theory
		      \begin{itemize}
		          \item Can give details about different types of interactions; important in understanding the nature of them
		          \item Not as cheap as DFT-D3
				  \item SAPT0 is somewhat cheap, but does not include intramonomer correlation
		      \end{itemize}
		  \end{itemize}
		\end{frame}

    \subsection{Subsection 2}

        \begin{frame}{Test 2}
            \begin{block}{Definition}
                Lorem ipsum dolor sit amet, consetetur sadipscing elitr, sed diam nonumy eirmod tempor invidunt ut labore et dolore magna aliquyam erat, sed diam voluptua. At vero eos et accusam et justo duo dolores et ea.
            \end{block}
        \end{frame}

        \begin{frame}
            \frametitle{Test 3}
            \begin{block}{Definition}
            \begin{itemize}
                \item Test1
                \item Test1
                \item Test1
            \end{itemize}
            \end{block}
        \end{frame}

   \subsection{Subsection 3}
   
        \begin{frame}{Test 2}
       	    Lorem ipsum dolor sit amet, consetetur sadipscing elitr, sed diam nonumy eirmod tempor invidunt ut labore et dolore magna aliquyam erat, sed diam voluptua. At vero eos et accusam et justo duo dolores et ea.

            \begin{align}
                R(\alpha) = \int \frac{1}{2} |y - f(x;\alpha)| dP(x,y) \leq R_{emp}(\alpha) + \epsilon(N, \delta, h)
            \end{align}   
        \end{frame}

   
        \begin{frame}{Test 2}
                Lorem ipsum dolor sit amet, consetetur sadipscing elitr, sed diam nonumy eirmod tempor invidunt ut labore et dolore magna aliquyam erat, sed diam voluptua. At vero eos et accusam et justo duo dolores et ea.
            \begin{block}{Definition}
                Lorem ipsum dolor sit amet, consetetur sadipscing elitr, sed diam nonumy eirmod tempor invidunt ut labore et dolore magna aliquyam erat, sed diam voluptua. At vero eos et accusam et justo duo dolores et ea.
            \end{block}
            \pause
            \begin{block}{Definition}
                Lorem ipsum dolor sit amet, consetetur sadipscing elitr, sed diam nonumy eirmod tempor invidunt ut labore et dolore magna aliquyam erat, sed diam voluptua. At vero eos et accusam et justo duo dolores et ea.
            \end{block}
        \end{frame}

        \begin{frame}{Test 3}
            \begin{block}{Normal block}
            block text
            \end{block}
            
            \begin{exampleblock}{Example block}
            block text
            \end{exampleblock}
            
            
            \begin{alertblock}{Alert block}
            block text
            \end{alertblock}
        \end{frame}

        \begin{frame}{Test 4}
            \begin{theorem}
            block text
            \end{theorem}
            
            \begin{proof}
            block text
            \end{proof}

        \end{frame}


        \begin{frame}{Test 5}

            \begin{tabular}{|c|c|c|}
            \hline
            \textbf{Test0} & \textbf{Test1} & \textbf{Test2} \\
            \hline
            Test1 & Test2 & Test3 \LaTeX  \\
            \hline
            Test4 & Test5& Test6\\
            \hline
            \end{tabular}
            
        \end{frame}

        \begin{frame}{Test 6}

            \begin{exampleblock}{Example block}
                \smallskip 

                \begin{tabular}{|c|c|c|}
                \hline
                \textbf{Test0} & \textbf{Test1} & \textbf{Test2} \\
                \hline
                Test1 & Test2 & Test3 \LaTeX  \\
                \hline
                Test4 & Test5& Test6\\
                \hline
                \end{tabular}
            \end{exampleblock}

            \begin{block}{Block}
                \smallskip 

                \begin{tabular}{|c|c|c|}
                \hline
                \textbf{Test0} & \textbf{Test1} & \textbf{Test2} \\
                \hline
                Test1 & Test2 & Test3 \LaTeX  \\
                \hline
                Test4 & Test5& Test6\\
                \hline
                \end{tabular}
            \end{block}
        \end{frame}

	 \begin{frame}{Test 7}
            
         	\begin{block}{Block A}
            
                \begin{description}[longest label]
                    \item[short] Short stuff
                    \item[long] Longer stuff
                    \item[longest label] Longest stuff (insert cat)
                \end{description}
            
                \begin{itemize}
                    \item item1
                    \item item2
                    \item item3
                \end{itemize}  
            \end{block}

        \end{frame}

        \begin{frame}[t]{Frame with Columns}
            \begin{columns}[t]
                \column{0.4\textwidth}

                \begin{block}{Block 1}
                    Text here
                \end{block}

                \column{0.4\textwidth}

                \begin{block}{Block 2}
                    More text here
                \end{block}
            \end{columns}
        \end{frame}

        \begin{frame}[t]{Frame without Columns}
            \leavevmode
            \begin{block}{Block}
                Even more text here
            \end{block}
        \end{frame}
\end{document}
